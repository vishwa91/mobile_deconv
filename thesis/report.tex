% Use the IITM Dissertion class
\documentclass[BTech]{iitmdiss}
\usepackage{times}
\usepackage{t1enc}

\usepackage{graphicx}
\usepackage{epstopdf}
\usepackage[hypertex]{hyperref} % hyperlinks for references.
\usepackage{amsmath} % easier math formulae, align, subequations \ldots

\begin{document}
%----------- Title page ------------------------------------------------
\title{Analyzing motion sensors on mobile platform for image restoration}
\author{Saragadam R V Vishwanath}
\date{MAY 2014}
\department{ELECTRICAL ENGINEERING}
\maketitle
\pagebreak

%----------- Certificate -----------------------------------------------
\certificate

\vspace*{0.5in}

\noindent This is to certify that the thesis titled {\bf Analyzing
 motion sensors on mobile platform for image restoration}, submitted by
  {\bf Saragadam Raja Venkata Vishwanath}, 
  to the Indian Institute of Technology, Madras, for
the award of the degree of {\bf Bachelor of Technology}, is a bona fide
record of the research work done by him under our supervision.  The
contents of this thesis, in full or in parts, have not been submitted
to any other Institute or University for the award of any degree or
diploma.

\vspace*{1.5in}

\begin{singlespacing}
\hspace*{-0.25in}
\parbox{2.5in}{
\noindent {\bf Prof. A. N. Rajagopalan} \\
\noindent Research Guide \\ 
\noindent Professor \\
\noindent Dept. of Electrical Engineering\\
\noindent IIT-Madras, 600 036 \\
} 
\hspace*{1.0in} 
\end{singlespacing}
\vspace*{0.25in}
\noindent Place: Chennai\\
Date: 12th May 2014

\pagebreak

%----------- Acknowledgement -------------------------------------------
\acknowledgements

I would like to thank all my friends and my family for supporting me 
through the project. I am indebted to my lab mates who have constantly 
helped me come over every obstacle in the project. I would also like to 
thank Prof. A N Rajagopalan for constantly following up with my project
and giving me valuable advice.

%----------- Abstract --------------------------------------------------
\abstract
We explore various image restoration and registration techniques using 
a mobile device. We present various methods for depth estimation of the 
scene, image registration for translation and rotation and image 
deconvolution. For this purpose, we have used a Nokia Lumia 520 mobile
device which features Windows Phone 8 operating system. 

%----------- Table of content ------------------------------------------
\begin{singlespace}
\tableofcontents
\thispagestyle{empty}

\listoftables
\addcontentsline{toc}{chapter}{LIST OF TABLES}
\listoffigures
\addcontentsline{toc}{chapter}{LIST OF FIGURES}
\end{singlespace}
\pagebreak

%----------- Abbrevations ----------------------------------------------
\abbreviations

\noindent 
\begin{tabbing}
xxxxxxxxxxx \= xxxxxxxxxxxxxxxxxxxxxxxxxxxxxxxxxxxxxxxxxxxxxxxx \kill
\textbf{IITM}   \> Indian Institute of Technology, Madras \\
\textbf{WP8}    \> Windows Phone 8 \\
\end{tabbing}

\pagebreak

%----------- Notations -------------------------------------------------
\chapter*{\centerline{NOTATION}}
\addcontentsline{toc}{chapter}{NOTATION}

\begin{singlespace}
\begin{tabbing}
xxxxxxxxxxx \= xxxxxxxxxxxxxxxxxxxxxxxxxxxxxxxxxxxxxxxxxxxxxxxx \kill
\textbf{$f$}    \> Latent image \\
\textbf{$g$}    \> Observer/blurred image \\
\textbf{$k$}    \> Blur kernel \\
\end{tabbing}
\end{singlespace}

\pagebreak

% Start numbering the pages from now.
\pagenumbering{arabic}

%----------- Introduction ----------------------------------------------
\chapter{INTRODUCTION}
\label{chap:intro}
There has been a recent shift in computation from traditional computers
to ubiquitous mobiles. With the ever increasing power of the mobiles, 
executing many of the image processing algorithms on mobile has been 
possible. Further, the many peripherals available for the mobile, like
the inertial sensors, controllable focus, TCP, bluetooth etc., 
computational photography is greatly advanced. 

To understand the potential of a mobile for computational photography, 
we choose to evaluate various algorithms on the mobile and evaluate the 
performance. For this purpose, we have chosen the Nokia Lumia 520 mobile
which features Windows Phone 8 operating system. 

The project report consists of the following part. The chapter on 
\textbf{THEORY} presents a brief background of the various algorithms
and the mathematics involved in the project. \textbf{DEVICE} gives an
in-depth account of the mobile platform, the application written for the
mobile platform and the host side software. \textbf{DEBLURRING} 
explains the implementation of the deblurring algorithm using the data
obtained from the mobile. \textbf{DEPTH ESTIMATION} has two parts, 
\emph{Depth estimation using motion blur} and \emph{Depth estimation 
using focus measure}. \textbf{IMAGE REGISTRATION} discusses about how
inertial sensor data can be used to reduce the search space of image 
registration process. 

The end of the project report has in-depth discussion regarding future
usage of the mobile application as a general prototyping platform and a
brief discussion of the \emph{github} version control system.

%----------- Basic Theory ----------------------------------------------
\chapter{BASIC THEORY}
\label{chap:basic_theory}
In this chapter, we look briefly at the back ground which is necessary 
to understand the forthecoming chapters. 
\end{document}
